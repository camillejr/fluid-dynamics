The material derivative is an operator defined as:

\begin{equation}
\frac{D}{D t} \equiv \frac{\partial}{\partial t} + \vec{\bm{V}} \cdot \nabla
\end{equation}

Or, expanding out the terms (in 3D case, where $\vec{\bm{V}} = \langle u, \upsilon, w \rangle$):

\begin{equation}
\frac{D}{D t} \equiv \frac{\partial}{\partial t} + u \frac{\partial}{\partial x} + \upsilon \frac{\partial}{\partial y} + w \frac{\partial}{\partial z}
\end{equation}

Material derivative is indeed a shorthand for writing a special sum of other operators and it has been created because this set is frequently used in fluid dynamics related equations. Writing it in short as $\frac{D}{D t}$ simply makes life easier.